\subsection{GULASCH SZEGEDINER ART}
%%%%%%%%%%%%%%%%%%%%%%%%%%%%%%
\begin{tabular}{ll}
\\
anspruch: & \grade{2.5} \\
zeitaufwand:& \grade{4}\\
\end{tabular}
\hspace{2cm}
\begin{tabular}{ll}
\\
kosten: & \grade{4} \\
veggi?& nein\\
\end{tabular}
\\ \\ \\
%%%%%%%%%%%%%%%%%%%%%%%%%%%%%%
\underline{\textbf{zutaten für 4 personen:}}
\begin{itemize}
 \item zwiebeln
 \item knoblauch
 \item 2 bis 3 paprika (farbe nach präferenz)
 \item 500 g rindfleisch (gulaschfleisch zb schulter)
 \item kleine dose sauerkraut (ca. 300 g)
 \item ggf 2 bis 3 kartoffeln (festkochend)
 \item ca. 2 el paprikamark (scharf)
 \item ca. 2 el tomatenmark
 \item rotwein
 \item ggf 1 bis 2 el mehl
\end{itemize}
öl, salz, pfeffer, paprika (edelsüß, scharf, smoked), chili, kümmel, lorbeer, wacholderbeeren\\ \\
%%%%%%%%%%%%%%%%%%%%%%%%%%%%%%%
zwiebeln, knoblauch, paprika und kartoffeln  schneiden. fleisch scharf anbraten. hier ggf das  mehl hinzufügen damit es später sämiger wird. fleisch aus der pfanne nehmen. zwiebeln glasig anbraten. knoblauch, paprika, kartoffeln, tomatenmark und paprikamark hinzugeben und mit anbraten. mit rotwein großzügig ablöschen (mind. 200 ml). fleisch hinzufügen und ggf wasser aufgießen bis gewünschte konsistenz erreicht ist. mit paprikapulver, kümmel, salz und pfeffer, 2 bis 3 lorbeerblätter und ca. 6 wacholderbeeren und chili abschmecken. köcheln lassen bis das fleisch weich wird. fertig! \\ \\
%%%%%%%%%%%%%%%%%%%%%%%%%%%%%%%
\textbf{dazu passt gut} 
\begin{itemize}
 \item rotwein
 \item saure sahne
 \item baguette oder ciabatta
\end{itemize}
\pagebreak