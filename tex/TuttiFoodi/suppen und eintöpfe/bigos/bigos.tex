\subsection{BIGOS A LA FAMILIE MASARCZYK}
%%%%%%%%%%%%%%%%%%%%%%%%%%%%%%
\begin{tabular}{ll}
\\
anspruch: & \grade{3} \\
zeitaufwand:& \grade{5}\\
\end{tabular}
\hspace{2cm}
\begin{tabular}{ll}
\\
kosten: & \grade{4} \\
veggi?& möglich\\
\end{tabular}
\\ \\ \\
%%%%%%%%%%%%%%%%%%%%%%%%%%%%%%
\underline{\textbf{zutaten für 6 personen:}}
\begin{itemize}
 \item ca. 5 zwiebeln (mind. 100 g)
 \item knoblauch
 \item ca. 140 g tomatenmark (3-fach)
 \item 1 kg weißkohl
 \item mind. 500 g champignons
 \item 1 große dose sauerkraut (850 g)
 \item 250 bis 500 g kassler (für veggi-varinate weglassen)
 \item 250 bis 500 g cabanossi (für veggi-varinate weglassen)
 \item 1 kg schweine(nacken/schulter)braten (für veggi-varinate weglassen)
\end{itemize}
öl, salz, pfeffer, 3 lorbeerblätter, mind. 6 wacholderbeeren, paprika edelsüß und scharf, ggf. chili \\ \\
%%%%%%%%%%%%%%%%%%%%%%%%%%%%%%%
zwiebeln, knoblauch, kassler und cabanossi schneiden. weißkohl hobeln. schweinenacken scharf anbraten und als braten im ofen zubereiten. kassler und cabanossi getrennt in öl abraten und wieder aus dem topf nehmen. kohl in topf leicht anbraten, salzen und mit ein wenig wasser köcheln lassen. sauerkraut in separatem topf kochen. nicht mit dem kohl zusammen! alles in einem topf mit viel tomatenmark, bratensaft, lorbeerblättern, wacholderbeeren und ggf. chili und paprika edelsüß würzen. mit salz und pfeffer abschmecken. fertig! \\ \\
%%%%%%%%%%%%%%%%%%%%%%%%%%%%%%%
\textbf{dazu passt gut} 
\begin{itemize}
 \item baguette oder ciabatta
\end{itemize}
\pagebreak