\subsection{NASI GORENG}
%%%%%%%%%%%%%%%%%%%%%%%%%%%%%%
\begin{tabular}{ll}
\\
anspruch: & \grade{3} \\
zeitaufwand:& \grade{3.5}\\
\end{tabular}
\hspace{2cm}
\begin{tabular}{ll}
\\
kosten: & \grade{3} \\
veggi?& möglich\\
\end{tabular}
\\ \\ \\
%%%%%%%%%%%%%%%%%%%%%%%%%%%%%%
\underline{\textbf{zutaten für 4 personen:}}
\begin{itemize}
 \item zwiebeln
 \item knoblauch
 \item ingwer (daumengroßes stück)
 \item 4 bis 6 eier 
 \item ca. 300 g tk erbsen
 \item ca. 300 g möhren
 \item 2 paprika
 \item 250 g pilze
 \item ggf. brokoli oder anderes gemüse der wahl
 \item 250 g reis
\end{itemize}
öl, sesamöl, salz, pfeffer, chiliflocken, cayennepfeffer, kreuzkümmel, koreander, ingwerpulver (ggf), curry oder kurkuma, sojasoße, reisessig, fischsoße\\ \\
%%%%%%%%%%%%%%%%%%%%%%%%%%%%%%%
reis kochen und kaltstellen. je nach geschmack mit 3 el reisessig vermischen. zwiebeln, ingwer, knoblauch und gemüse schneiden. zwiebeln in öl und ca. 3 el sesamöl in einer großen pfanne glasig braten.  Ingwer und chiliflocken dazu geben und auf mittlerer hitze weiter braten. knoblauch dazu pressen. kreuzkümmel, koreander und curry zu der masse geben und mit anbraten. zuerst möhren und tk erbsen anbraten, später paprika und pilze dazu. masse leicht salzen. nicht zu stark wegen der sojasoße. wenn gemüse noch knackig ist den abgekühlen reis dazu geben und durchmischen. reis kurz anbraten dann auf mittlere hitze runter stellen. großzügig mit sojasoße und 1 bis 2 el  fischsoße übergießen. für veggi-variante fischsoße weglassen. mit salz, pfeffer, sojasoße und gewürzen abschmecken. eier zur pfanne schlagen und untermengen. fertig!\\ \\
\textbf{dazu passt gut}
\begin{itemize}
    \item sriracha-sauce
    \item röstzwiebeln
\end{itemize}
\pagebreak