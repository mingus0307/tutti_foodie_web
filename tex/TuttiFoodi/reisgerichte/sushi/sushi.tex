\subsection{SUSHI}
%%%%%%%%%%%%%%%%%%%%%%%%%%%%%%
\begin{tabular}{ll}
\\
anspruch: & \grade{3} \\
zeitaufwand:& \grade{4}\\
\end{tabular}
\hspace{2cm}
\begin{tabular}{ll}
\\
kosten: & \grade{5} \\
veggi?& möglich\\
\end{tabular}
\\ \\ \\
%%%%%%%%%%%%%%%%%%%%%%%%%%%%%%
\underline{\textbf{zutaten für 4 personen:}}
\begin{itemize}
 \item 250 g shushireis oder milchreis (günstiger, aber klappt genauso)
 \item ca. 8 noriblätter
 \item 2 paprika (farbe nach präferenz)
 \item 1/2 gurke
 \item 1 bis 2 möhren
 \item 2 avocados
 \item ggf. pilze
 \item 1 pck räucherlachs (ca. 250g) (für veggi-variante einfach weglassen)
 \item 3 el reisessig
 \item ggf. sesamkörner
\end{itemize}
sojasoße, wasabipaste\\ \\
%%%%%%%%%%%%%%%%%%%%%%%%%%%%%%%
reis nach packungsanleitung kochen. reisessig hinzufügen und abkühlen lassen. gemüse in streifen schneiden. jetzt wird gerollt. eine schüssel wasser bereitstellen. ein noriblatt mit der rauen seite zu dir leicht mit wasser anfeuchten. reis darauf verteilen und fest andrücken. die obersten 2 cm ausgesparen. gemüse in der mitte als streifen verteilen. je weniger desto einfacher zu rollen. sesamkörner darüber verteilen. von unten bis hinter den gemüsestreifen und dann bis oben einrollen. den freien streifen nun auch mit wasser einstreichen und die rolle fest verkleben. rolle in kleine teile schneiden. mit sojasoße und wasabi servieren. fertig! \\ \\
%%%%%%%%%%%%%%%%%%%%%%%%%%%%%%%
\textbf{dazu passt gut} 
\begin{itemize}
 \item gemischter salat (nicht verrollte gemüsereste können hier gut verwertet werden)
\end{itemize}
\pagebreak