\subsection{SCHNITZEL MIT GEMÜSE UND KÜMMELKARTOFFELN}
%%%%%%%%%%%%%%%%%%%%%%%%%%%%%%
\begin{tabular}{ll}
\\
anspruch: & \grade{2} \\
zeitaufwand:& \grade{2.5}\\
\end{tabular}
\hspace{2cm}
\begin{tabular}{ll}
\\
kosten: & \grade{4} \\
veggi?& möglich\\
\end{tabular}
\\ \\ \\
%%%%%%%%%%%%%%%%%%%%%%%%%%%%%%
\underline{\textbf{zutaten für 2 personen (schnelle veggi variante):}}
\begin{itemize}
 \item 400 g festkochende kartoffeln
 \item 2 pck veggi schnitzel zb. valess
 \item 450 g gemüse zb. tk brokoli, tk blumenkohl oder spargel
 \item 1 pck sauce hollandaise zb. thomi
\end{itemize}
\underline{\textbf{für 4 kalbsschnitzel:}}
\begin{itemize}
 \item 4 kalbsschnitzel
 \item 3 eier 
 \item semmelbrösel
 \item mehl
\end{itemize}
\underline{\textbf{zutaten für sauce holondaise:}}
\begin{itemize}
 \item 2 el butter
 \item 1 el mehl
 \item ca. 250 ml (hafer)milch
\end{itemize}
öl, butter, salz, pfeffer, kümmel (ganz), muskatnuss\\ \\
%%%%%%%%%%%%%%%%%%%%%%%%%%%%%%%
kartoffeln schälen und schndeiden. mit kümmel und salz kochen. gemüse salzen und dämpfen oder kochen. fleisch klopfen. separate schüsseln mit mehl, eiern und semmelbröseln füllen. eier und semmel salzen und pfeffern. eier quirlen. fleisch panieren (1. mehl, 2. eier, 3. semmel). bei bedarf 2x mit eiern und semmel pannieren. öl erhitzen und 1 el butter hinzufügen. schnitzel darin braten. für holondaise butter schmelzen und mehl hinzufügen. kurz leicht anbraten. mit milch ablöschen bis gewünschte konsistenz erreicht ist. salz, pfeffer und muskatnuss (gerieben) hinzufügen. zusammen servieren. fertig! \\ \\
%%%%%%%%%%%%%%%%%%%%%%%%%%%%%%%
\pagebreak
%%%%%%%%%%%%%%%%%%%%%%%%%%%%%%
