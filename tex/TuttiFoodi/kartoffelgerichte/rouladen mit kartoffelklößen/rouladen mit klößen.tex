\subsection{ROULADEN MIT KARTOFFELKLÖSSEN}
%%%%%%%%%%%%%%%%%%%%%%%%%%%%%%
\begin{tabular}{ll}
\\
anspruch: & \grade{2.5} \\
zeitaufwand:& \grade{4}\\
\end{tabular}
\hspace{2cm}
\begin{tabular}{ll}
\\
kosten: & \grade{4} \\
veggi?& nein\\
\end{tabular}
\\ \\ \\
%%%%%%%%%%%%%%%%%%%%%%%%%%%%%%
\underline{\textbf{zutaten für 2 personen:}}
\begin{itemize}
 \item 4 rouladen
 \item schinken (z.B. schwarzwälder oder bacon)
 \item gewürzgurken
 \item zwiebeln
 \item knoblauch
 \item rotwein
 \item senf
 \item kartoffelklöße (aus kartoffelmehl und kartoffelstärke 50:50)
 \item 700 g rotkohl (1 glas)
 \item apfel
\end{itemize}
öl, salz, pfeffer, zucker, (gemüsebrühe)\\ \\
%%%%%%%%%%%%%%%%%%%%%%%%%%%%%%%
rouladen rauslegen damit sie auf raumtemperatur kommen.zwiebeln und knoblauch würfeln. halbe zwiebelringe schneiden und goldbraun anbraten. rouladen mit senf und speck drauf legen. gurken würfeln und mit zwiebeln auf den rouladen verteilen. mit pfeffer und salz würzen. rouladen rollen. rouladen scharf anbraten und aus der pfanne nehmen. zwiebeln und knoblauch anbraten. mit etwas tomatenmark anbraten und mit rotwein ablöschen. mit etwas brühe oder wasser aufgießen. rouladen in die soße legen. mit lorberblatt, wacholderbeeren, senf, salz und pfeffer würzen. mindestens 1.5 h auf niedriger stufe köcheln. kartoffelklöße nach anleitung zubereiten. einen apfel und eine zwiebel vierteln. rotkohl mit zwiebeln und rotkohl köcheln lassen. alles zusammen servieren. fertig! \\ \\
%%%%%%%%%%%%%%%%%%%%%%%%%%%%%%%
kleiner tipp: von den klößen und rouladen kann man auch mehr machen und einfrieren. \\ \\
\textbf{dazu passt gut} 
\begin{itemize}
 \item rotwein
\end{itemize}
\pagebreak