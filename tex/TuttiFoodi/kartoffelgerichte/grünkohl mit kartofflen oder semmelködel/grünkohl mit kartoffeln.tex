\subsection{GRÜNKOHL MIT KARTOFFELN ODER SEMMELKNÖDELN}
%%%%%%%%%%%%%%%%%%%%%%%%%%%%%%
\begin{tabular}{ll}
\\
anspruch: & \grade{1} \\
zeitaufwand:& \grade{4}\\
\end{tabular}
\hspace{2cm}
\begin{tabular}{ll}
\\
kosten: & \grade{3} \\
veggi?& nein\\
\end{tabular}
\\ \\ \\
%%%%%%%%%%%%%%%%%%%%%%%%%%%%%%
\underline{\textbf{zutaten für 6 personen:}}
\begin{itemize}
 \item 1.5 bis 2 kg tk grünkohl
 \item ca. 4 zwiebeln (je nach größe)
 \item knoblauch
 \item 300 g bzw. 1 pck. mettenden
 \item optional 500 g kartofflen festkochend
 \item optional 250 g kassler 
\end{itemize}
%%%%%%%%%%%%%%%%%%%%%%%%%%%%%%
\underline{\textbf{zutaten für 2 semmelknödel:}}
\begin{itemize}
 \item 300 g mehl 
 \item 40 g semmelbrösel oder 1 altes brötchen
 \item 2 tl backpulver
 \item 1 ei
 \item ca. 500 ml milch
\end{itemize}
salz, pfeffer, ca. 5 wacholderbeeren, 3 lorbeerblätter, 6 pimentkörner, 4 bis 5 el senf\\ \\
%%%%%%%%%%%%%%%%%%%%%%%%%%%%%%%
zwiebeln und knoblauch grob würfeln und anbraten. grünkohl mit etwas wasser hinzufügen. kassler und mettenden schneiden und hinzufügen. salz, pfeffer, senf und gewürze hinzufügen. für mind. 2 h köcheln lassen. kartoffeln kochen und servieren. fertig!\\ \\
für knödel großen topf mit salzwasser um kochen bringen. mehl, semmelbrösel und backpulver vermischen. 1 ei hinzufügen und mit milch verkneten. 2 knödel daraus formen. knödel solange im kochwasser drehen bis er schwimmt. deckel drauf und 15 min kochen. deckel dabei nicht abnehmen. knödel wenden nochmal 15 min kochen. deckel wieder nicht lüften. fertig!
\pagebreak